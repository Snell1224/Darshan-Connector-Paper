\begin{abstract}
%Allowing for further insight into I/O behavior and patterns has become increasingly important. The I/O behavior depends on a large number of components such as the applications access pattern, the computing system architecture, I/O libraries, file system, mode of access, data size, and the storage configuration and layout. Changes in these components can create high variations in I/O performance and behavior which creates a lack of understanding about the application I/O and very difficult to identify which of these components are affecting I/O variability and behavior.  
%%and most of the time strong correlations of I/O analyses across different applications are required to identify a possible solution. 
%This paper introduces a unique framework that provides low latency monitoring of I/O event data during run time. This is done through the implementation of a system-level infrastructure that continuously collects I/O application data from an existing I/O characterization tool. This allows insights into the I/O application behavior and the components affecting it through analyses and visualizations.
%The framework allows users to better understand throughput for system-specific behaviors, variations of I/O performance of similar applications across a system and identify correlations between I/O and system behavior. 
%%This paper demonstrates the implementation and design of this framework and how it can be used on various HPC applications.
	Periodic capture of comprehensive, usable I/O data for scientific applications requires an easy-to-use technique to record information throughout the execution without causing substantial performance effects. %This technique will facilitate capturing I/O performance and behavior (which can create a lack of understanding about an HPC application) and \emph{help} identify which components affect I/O variability and behavior. 
	In this paper, we introduce a unique framework that provides low latency monitoring of I/O event data during run time. We implement a system-level infrastructure that continuously collects I/O application data from an existing I/O characterization tool to enable insights into the I/O application behavior and the components affecting it through analyses and visualizations. In this effort, we evaluate our framework by analyzing sampled I/O data captured from two HPC benchmark applications to understand the I/O behavior during the execution life of the applications. The result shows the utility of capturing I/O application performance and behavior.
\end{abstract}
