\section{Darshan LDMS Integration}
\label{sec:DarshanLDMSIntegration}

%Therefore, this paper provides an existing system level system
Darshan collects I/O application data for post-run analysis, and LDMS has a low overhead sampling capability supported by fast storage and a modern web interface. We chose to enhance both Darshan and LDMS by adding a sampling capability that acts as a connector between these two applications to support I/O runtime data sampling and visualization. Here is a high-level overview of the design and implementation of the \Darshan{} and the components used to create this infrastructure:
\begin{itemize}
	\item The I/O characterization tool, Darshan, to collect application I/O behavior and patterns. 
	%However, this tool does not report the \emph{absolute timestamp} so modifications to the code were made to expose this data.
	\item LDMS to provide and transport live run time data feed about application I/O events.~\cite{ldmsgithubwiki}
	\item DSOS to store and query large amounts of data generated on a production HPC system.~\cite{sosgithub}
	\item HPC Web Services to present run time I/O data to enable the user to create new meaningful analyses~\cite{ClusterAV}.
\end{itemize}

%This approach will provide run time insights about application I/O by using the following tools:
%\begin{enumerate}
%	\item The I/O characterization tool, Darshan, to collect application I/O behavior and patterns. However, this tool does not report the \emph{absolute timestamp} so modifications to the code were made to expose this data.
%	\item LDMS to provide and transport live run time data feed about application I/O events.~\cite{ldmsgithubwiki}
%	\item DSOS to store and query large amounts of data generated on a production HPC system.~\cite{sosgithub}
%	\item HPC Web Services to present run time I/O data. The timeseries data will enable the user to identify when a variability occurs as well as create new meaningful analyses.~\cite{ClusterAV}
%\end{enumerate}

%Darshan is used to tune applications for increased scientific productivity or performance and is suitable for full time deployment for workload characterization of large systems~\cite{darshan-webpage}. It provides detailed statistics about various level file accesses made by MPI and non-MPI applications which can be enabled or disabled as desired. These levels include POSIX, STDIO, LUSTRE and MDHIM for non-MPI applications and MPI-IO, HDF5 and some PnetCDF for MPI applications~\cite{darshan-runtime}. This functionality provides users with a summary of I/O behavior and patterns from an application run but it does so post-run. Therefore, it does not allow insights into \emph{run time} I/O behavior and patterns which makes it nearly impossible to identify the root cause(s) of I/O variability and when this occurs. 

LDMS is used to efficiently collect and transport scalable \emph{synchronous} and \emph{event-based} data with low-overhead. Two key functionalities the \Darshan{} will leverage in order to create the \connector{} are the LDMS publish-subscribe bus and LDMS transport~\cite{ldmsgithub}. In this work, we enhanced LDMS to support application I/O data injection and store it to DSOS, then modified Darshan to expose the \emph{absolute timestamp} to publish run time I/O events for each rank to \emph{LDMS Streams}. This integration will be described in detail later on in paper.   

%LDMS is used to efficiently collect and transport scalable \emph{synchronous} and \emph{event-based} data with low-overhead. Two key functionalities it has that will be leveraged in the \Darshan and create the \connector are the \emph{LDMS Streams} and transport~\cite{ldmsgithub}. In the \Darshan, the LDMS was enhanced to support application I/O data injection and store to DSOS while Darshan was modified to expose the \emph{absolute timestamp} and publish run time I/O events for each rank to the \emph{LDMS Streams}. This integration will be described in detail later on.   



%Implementing LDMS into Darshan will provide timestamped I/O event data during an application run. This data will provide any users, application developers or researchers to further their analyses, better understand when an I/O variability occurs and can identify any correlations between the I/O performance and file system, network congestion or system contention.  

\emph{DSOS} will enable the ability to query the timestamped application I/O data through a variety of APIs while Grafana will provide a front-end interface for visualizing the stored data that has been queried and analyzed using Python based modules on the back-end. With these tools, users can view, edit and share analyses of the data as well as create new meaningful analyses. 

%The implementation of LDMS into Darshan along with the storage, analysis and visualization components that make up this design will provide detailed insights into the I/O behavior and patterns during run time. This insight will allow users and researchers to better understand how application I/O variability correlates with overall system behavior (e.g. file system, network congestion, etc.) how the time of day affects the I/O performance, how the I/O pattern within a run affects the I/O performance of read and write and why the read and write I/O performance patterns are different and independent of each other. Further, the occurrence of any I/O variability can be identified during runtime.

%\RED{\begin{itemize}
		%    \item Address current I/O performance analysis and implications.
		%    \subitem Lesson Learned: Different applications experience high and low performance variations at separate and disjoint time periods.These times periods are shared across different applications and clusters. 
		%    \subitem How LDMS will help: It is suggested that a simple I/O monitoring data collection from Darshan will be useful in identifying these time periods. LDMS will be able to provide the run time I/O monitoring data collection which will help them in identifying these time periods. 
		%    \subitem Lesson Learned: Clusters running on weekends observe some of the highest performance variations. It is assumed that this could be from I/O intensive application runs during the weekends. 
		%    \subitem How LDMS will help: The I/O timeseries data generated by LDMS will provide NERSC researchers with new data they can utilize to further their research.
		%    \item Benefits of having time series data:
		%    \subitem Better understand how application I/O variability correlates with overall system behavior
		%    \subitem Understand how time of day affects the I/O performance
		%    \subitem Understand how the I/O pattern within a run affects the I/O performance of read and write. E.g. perhaps an application that has nodes writing data at disjoint times have less I/O performance variation than one that has all nodes writing data at the same time.
		%    \subitem Understand why read and write I/O performance patterns are different and independent of each other. E.g. read and write may exhibit different behavior during the run, which affects the performance of each in unique ways
		
		%\end{itemize}
		%}
	
	%\subsection{Darshan and LDMS Overview}
	%\RED{ \begin{itemize}
			%    \item Explain how Darshan works. What is it, who created it, what does it do %exactly? What areas does it lack? (i.e. no timestamps)
			%    \item Explain how LDMS works (high level overview -- same as Darshan). What capabilities does it have that we will be applying to Darshan to transport the data.
			%    \item Explain how SOS works and why we are using it.
			%    \item Explain how Grafana works with SOS and why we are using it. 
			%\end{itemize}
			%}

