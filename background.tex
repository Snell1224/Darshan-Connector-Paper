\section{Background}
\label{sec:background}
Darshan is a lightweight I/O characterization tool that captures I/O access pattern information from HPC applications.~\cite{Darshan} This will be described in detail in Section III. This I/O characterization tool will be used for tracing and collecting detailed I/O event data.

The Lightweight Distributed Metric Service (LDMS) is a low-overhead system that collects and transports HPC data for OVIS via \emph{samplers} and \emph{plugins}.~\cite{ovisweb} OVIS which is a modular system for collecting, analyzing, storing, transporting and visualization HPC data in order to provide further insights into the system state, resource utilization and performance. A \emph{sampler} is a type of daemon that collects the data while the \emph{plugin} determines the type of data to be sampled, aggregated or stored and a \emph{sampler} is the set configuration for collecting the data specified by the plugin. There are a variety of samplers and other plugins that can be written for the LDMS API (in C). The system state insights are achieved by LDMS's \emph{absolute-timestamp} view of system conditions through multi-hop \emph{aggregation} and the LDMS Transport. An LDMS daemon, \emph{LDMSD}, aggregator supports multiple levels, networks and security domains so data can be sent to various locations. Additional functionalities, such as the \emph{LDMS Streams API}, has been developed to allow for aggregation of event-based application data. This framework utilizes \emph{LDMS Streams API} to collect Darshan's I/O event data during execution time and store the \emph{timestamped} data to a database.

The Distributed Scalable Object Store (DSOS) is a storage database designed to efficiently manage large volumes of HPC data~\cite{sosgithub}. It supports high data injection rates, has an enhanced query performance and flexible storage management. DSOS has a command line interface for data interaction and has various program API's such as Python, C and C++. A DSOS cluster consists of multiple instances of DSOS daemons, \emph{dsosd}, that run on multiple storage servers on a single cluster. The DSOS Client API can perform parallel queries to all \emph{dsosd} in a DSOS cluster. The results of the queried data are then returned in parallel and sorted based on the index selected by the user. This database and it's Python API are used in this framework for storing and querying the I/O event data. 

The HPC Web Services is an analysis and visualization infrastructure ~\cite{ClusterAV}, that integrates an open-source web application, Grafana~\cite{grafana-website}, with a custom back-end web framework (Django)  which calls python modules for analysis and visualization of HPC data. Grafana is an open-source visualization tool tailored towards time-series data from various database sources. Grafana provides charts, graphs, tables, etc. for viewing and analyzing queried data in real time. Using a custom DSOS-Grafana API, the python analysis modules to be used can be specified in a Grafana query. Once specified, that python analysis transforms any data queried from the dashboard before returning the data to Grafana. Grafana enables a wide variety of visualization options for the data and allows users to save and share those visualizations to others. This framework will leverage the HPC Web Services for run time analyses and visualizations of the I/O event data.
