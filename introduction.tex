\section{Introduction}
As more scientific I/O applications are developed and used, the need for
improved fidelity and throughput of these applications is more pressing than ever. 
Much design effort and investments are being put into improving not only
I/O applications but also the system components. Being able to
identify, predict and analyze I/O behaviors are critical to ensuring
parallel storage systems are utilized efficiently~\cite{costa2021}. 

%However, I/O
%performance continues to show high variations on large-scale
%production conditions in many cases~\cite{costa2021}.
%\RED{Cite  NE}. %Some of these cases include running applications on clusters
       %during the weekend, separate and disjoint time zones and read
       %I/O's of runs within the same cluster.

Variations in I/O can be caused by the system usage and behavior such as the file system (e.g. buffering, file transfer,
interrupt handling), network congestion, data set size and storage, system resource contentions,
or by the access patterns of the application itself~\cite{I/O-performance-variation}.
%Providing users, developerd, and system administratior with visual insight about the I/O behavior combined with other resources information during the runtime of the application will assest them in determining the root cause of I/O related problems. 
This variation
makes it difficult to determine the root cause of I/O related problems
and have a thorough understanding of throughput for system-specific
behaviors and I/O performance in similar applications across a
system. 
%Further, not knowing the origin of such variations will
%irectly affect the user and developer as unwanted time, effort and
%investment will need to be put into solving the issue.

Generally, the I/O performance is analyzed post-run by application
developers, researchers and users in the form of regression testing or
other I/O characterization tools that capture the applications I/O
behavior. An example of one of these tools is \emph{Darshan}, which
monitors and captures I/O information on access patterns from HPC
applications~\cite{Darshan}.
% Detailed information will be covered in the \emph{Approach} section.
Efforts to identify the origin of I/O performance given by these tools
usually come from any identified correlation between analyses of
various applications runs or the time in which these applications were
tested. However, this approach does not provide the ability to know
\textbf{when} an I/O performance variation occurs during an
application run and, if the developer or user wishes to, identify any
correlations between the file system, network congestion or resource
contentions and the I/O performance.

%However, this analysis approach does not take into account the file system, network congestion, system resource contentions and other component's affect on the I/O performance. In order to make these associations, the \emph{absolute timestamps} is required  which the post-run approach and I/O characterization tools usually lack.

Execution logs that provide \emph{absolute timestamps}
(e.g. timeseries) enable users and developers to perform temporal
performance analysis, and better understand how these changing
components affect the I/O performance variation as well as provide
further insight into the application I/O pattern. Therefore, we
introduce our \Darshan{} approach that incorporates the \emph{absolute
  timestamps} to provide a run time set of application I/O data. This
is a work in progress paper and will cover the following goals:

\begin{itemize}
	\item Describe the approach used to expose absolute timestamp
          data from an existing I/O characterization tool;
       	\item Provide a high level overview of the implementation
          process and other tools used to collect application I/O data
          during run time;
  	\item Demonstrate use cases of the \connector{} for two
          applications with distinct I/O behavior on a production HPC
          system;
        \item Utilize Darshan LDMS data to identify and better
          understand any root cause(s) of application I/O performance
          variation run time;
	\item Present how this new approach can be integrated with
          other tools to benefits users to collect and assist in the
          detection of application I/O performance variances across
          multiple applications.
\end{itemize}

%Section \ref{sec:background} presents the background and motivation
%for this paper, Section \ref{sec:DarshanLDMSIntegration} presents the
%approach to design the Darshan LDMS Integration and collect the
%absolute timestamps, Section \ref{sec:integration} depicts the tools
%integrated to our approach , Section \ref{sec:methodology} presents
%the experimental methodology to generated our results (Section
%\ref{sec:results}). Section \ref{sec:rw} presents the related works,
%and Section \ref{sec:conclusion} presents the conclusion and future
%works.
