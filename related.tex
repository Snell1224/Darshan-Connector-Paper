\section{Related Work}
\label{sec:rw}
Extensive work has been performed to improve I/O performance and behavior. 
The PASSION Runtime Library for parallel I/O proposed by Syracuse 
University~\cite{RuntimeLibrary-ParallelI/O} works to optimize I/O intensive 
applications through Data Prefetching and Data Sieving. The authors of "IOPin: Runtime 
Profiling of Parallel I/O in HPC Systems" propose dynamic instrumentation to 
show the interactions between a parallel I/O application and the file 
system~\cite{HPC-IO-Runtime}. "Design and Implementation of a Parallel I/O 
Runtime System for Irregular Applications"~\cite{RuntimeLibrary-IrregularApps} 
presents two different collective I/O techniques for improving I/O performance. 

Darshan was our preferred I/O characterization tool because the Darshan's eXtended 
Tracing (DXT)~\cite{darshan-runtime} instrumentation module provides high-fidelity 
traces for an application's I/O workload vs Darshan's traditional I/O summary 
data~\cite{darshan-runtime}. It also collects timestamped data which made it 
possible to expose the \emph{absolute timestamp} for collecting runtime timeseries data. 

Other open-source I/O tools that collect runtime timeseries data do exist. 
The linux command, iostat~\cite{iostat} collects system I/O device statistics 
and generates reports about the I/O loads between physical disks. The ioprof~\cite{ioprof} 
tool provides insights into I/O workloads. However, these tools do not provide the 
extensive I/O tracing capability (e.g. detailed statistics of individual I/O operations) 
that is provided by this framework. 

This work differs from these approaches because we \emph{leverage and enhance} 
existing applications and tools to design an infrastructure that creates runtime 
analyses and visualizations from detailed traces of application I/O events during 
execution time. The \Darshan{} integrates LDMS's \emph{
absolute timestamped} data collection and storage capabilities~\cite{ldmsgithub} 
with Darshan~\cite{darshan-webpage} to collect runtime application I/O data. 
Further, our choice of the DSOS database for storage of our event-based application
I/O data enables efficient queries of large volumes of data as well as python 
analysis modules and an open-source web application for runtime analyses and 
visualizations.

