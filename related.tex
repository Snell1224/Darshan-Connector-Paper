\section{Related Work}
Extensive research has already been proposed in literature to provide further insights into I/O behavior. The PASSION Runtime Library for parallel I/O proposed by Syracuse University~\cite{RuntimeLibrary-ParallelI/O} that optimizes I/O intensive applications through Data Prefetching and Data Sieving, the IOPin: Runtime Profiling of Parallel I/O in HPC
Systems proposes a dynamic instrumentation to show the interactions from a parallel I/O application to the file system~\cite{HPC-IO-Runtime} and Design and Implementation of a Parallel I/O Runtime System for Irregular Applications~\cite{RuntimeLibrary-IrregularApps} that proposes two different collective I/O techniques for improving I/O performance.

Darshan was the preferred I/O characterization tool because they had their Darshan's eXtended Tracing (DXT)~\cite{darshan-runtime} instrumentation module that provides high-fidelity traces for an application's I/O workload vs Darshan's traditional I/O summary data~\cite{darshan-runtime}. Other open-source I/O tools that we have come across or are aware of do not have this extensive I/O tracing capability which is leveraged in this work.

This work in progress differs from these approaches because we \emph{leverage and enhance} existing applications and tools to design an infrastructure that creates runtime analyses and visualizations from detailed traces of application I/O events during execution time. The \Darshan integrates LDMS's \emph{time stamped} data collection and storage capabilities~\cite{ldmsgithub} with Darshan~\cite{darshan-webpage} to collect runtime application I/O data. Further, a database is implemented to allow for efficient queries of large volumes of data as well as python analysis modules and an open-source web application for runtime analyses and visualizations.